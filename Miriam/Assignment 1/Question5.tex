\section*{Question 5}
\subsection*{a)}
% How many process instances and events are in this event log? What is the median
% number of events in each trace and what is the average duration of them?
The process has 150370 cases and 561470 events. Median number of events is 5.
The average duration is 48.8 weeks.

\subsection*{b)}
% What will be the Disco process model for this event log when you set Activities
% slider to 80% and Paths slider to 10%? Interpret the (self-loop) edge from
% �Payment� to itself. How many times and for how many process instances this
% behavior happens?

\includegraphics[width=0.8\textwidth]{Question5b.jpg}

The loop tells us, that a part of all people had to pay at least two times. If
you check the log data, you can see, that mostly it happens in the next 30 days.
Maybe they had a plan how to pay over a few weeks or had to pay penaltys.

It happen 4306 and for 4014 cases. So there are cases, where it happens more
than one time. You also can see, that it happens at most 14 times.

\subsection*{c)}
%Using Disco, analyze the time distribution of events over the time covered by this
%log? What is your interpretation of the distribution of events over the time?
% Do you see any remarkable patterns in the distribution of events (drifts, repeating
%behavior, etc.)?
\includegraphics[width=0.8\textwidth]{Question5cDist.jpg}

You can see that the distribution is going up and down a little bit, but there
are 10 peaks to see where more happens. In the end activity gets
lower and the distance between the 6th and 7th peak is higher than between the
others.
It is always around the typical paydays, so probably a lot of people then have
the money to pay the fine.

The peakes are on:
06.04.2002 (Sa), 06.01.2003 (Mo), 05.01.2004 (Mo), 24.12.2004 (Fr), 20.02.2006 (Mo),
18.02.2007 (Su), 27.03.2009 (Fr), 08.10.2010 (Fr), 22.03.2012 (Thu), 19.04.2013
(Fr)

\subsection*{d)}
% How many variants are in this event log? What is the size (number of process
% instances) of the third most frequent variant? Also, explain what happens in this
% variant (report the sequence of activities).
There are 231 variants. The third most frequent variant has 20385 instances. It
just contains the behaviour \textbf{create} and \textbf{send fine}. Nothing more
happens there.

\subsection*{e)}
% Filter the event log using Disco while keeping 50% of the most common variants of
% process instances that finished until 01.01.2012 12: 12. What happens to the
% median and the average of case duration compared to the whole event log. Please
% also explain the difference between the median and the average (mean) of case
% duration.
It is just possible to $43\%$ or $56\%$. The average case duration shorts to
45.2 weeks from 48.8 weeks and the median to 20.9 weeks from 28.3 weeks in both
cases.
So the most common cases are in average faster finished than all cases in
average.

The median is the instance in the middel. So if you write down all instances
sorted, it is the middel one. The average is the sum of all instances
divided by the number of instances. The average can change a lot for big or
extree small oultiers. The median shows more a real duration in the middle of
all instance durations.

\subsection*{f)}
% Discover the dotted chart view of the event log and interpret it using ProM. Adjust
% the Dotted chart settings in a way to answer question c. Are there any interesting
% (or odd) patterns in the dotted chart view that could explain the patterns found
% when answering question c?
\includegraphics[width=\textwidth]{Question5f.jpg}

In the first dotted chart you see when which case is active.

\includegraphics[width=\textwidth]{Question5fcZoomedIn.jpg}

For interpreting the dot chart for the question c) I changed the y-axis to the
event names so I can see when which events happening and also coloured the
events.

I also zoomed in (not to see in the screenshot) that I can see the months of the
years better. So I checked the dates of the peaks.

You can see that \textbf{Send for Credit Collection} is strongly connected to
the peakes to see in c). 

Payment happens mostly always, but still it is a little bit bundled at the
peaks.

Furthermore I would say, that in disco it is more easy to see when peaks happen,
but in prom better to see what is happen on the peak days.

\subsection*{g)}
% Filter the event log using �filter log using simple heuristics� and then apply the
% Alpha algorithm (Alpha ++) on it. Filter the log appropriately. Show some insights
% that can only be seen after filtering.

\includegraphics[width=\textwidth]{Question5g.jpg}

If you apply the alpha-algoirthm on the not filtered data you can not see so
clear, that in the resulting chart the payment can happen always and also
infinite often. It sounds weird, because you do not expect someone paying before
he gets the fine, but looking at the data it happens.

I also filtered with different settings
\begin{enumerate}
  \item 90\% for create fine and complete
\end{enumerate}

\subsection*{h)}
% Use the filtered event log in the previous question and this time use it as input for
% Disco. Which parts of the process are time consuming for most of the process
% instances? Answer this by interpreting the Disco model.

\includegraphics[width=\textwidth]{Question5h.jpg}

84days is the median for send fine. I chose the median, because then you get a
better idea what happens most of the times. The biggest mean is 23.3 weeks
between add penalty and payment.

What you also can see, is that after 60days always a penalty is added.
